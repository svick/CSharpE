\chapter{Attachments}

\section{The code}

The attached code is a snapshot of the repository \url{https://github.com/svick/CSharpE} and contains all the code of the CSharpE project, namely:

\begin{itemize}
\item The \cs{doc} directory contains the source of this document.

\item The \cs{example} directory contains a small project demonstrating the usage of several example extensions.

\item The \cs{src} directory contains the implementation of CSharpE. The directory structure corresponds to the projects outlined in Chapter~\ref{projects}, with few additions:

\begin{itemize}
\item The \cs{Extensions} subdirectory contains example extensions. (See Section \ref{example-extensions}.)

\item The \cs{SyntaxFactory} subdirectory contains a helper application that uses \cs{CSharpE.Syntax} to generate the \cs{SyntaxFactory} class.
\end{itemize}

\item The \cs{test} directory contains tests verifying functionality of CSharpE.

\end{itemize}

\subsection{Prerequisites}

Before using the code, it is required to have Visual Studio 2017 version 15.9 installed. All editions are supported, including the Community edition.

\todo{change to VS 2019; workloads and additional components}

\subsection{Building}

Building the code is not necessary in order to use CSharpE, because all the required assets can be downloaded from the Internet. Specifically, the Visual Studio extension is available on the Visual Studio Marketplace and NuGet packages for the MSBuild extension and example CSharpE extensions are available on NuGet.org, as is described in more detail in the following sections.

In order to build the code:

\begin{enumerate}
\item Open the solution \cs{CSharpE.sln} in Visual Studio.
\item If you are building CSharpE for the first time:

\begin{enumerate}
\item Change the target framework of the project \cs{Transform.VisualStudio} to .NET Framework 4.7.2.
\item Build the solution.
\item Change the target framework of \cs{Transform.VisualStudio} back to .NET Framework 4.6.
\end{enumerate}

Without this step, \cs{IgnoreAccessChecksToGenerar} will not work properly. \cite{iactg-net46}

\todo{check if the workaround mentioned in the issue works}

\item Build the solution. This will produce the VSIX file that contains the Visual Studio extension.

\item In order to produce NuGet packages for the MSBuild extension and example CSharpE extensions, you can execute \cs{dotnet pack} from the root of the repository (add \cs{-c Release} to the command to build the packages in Release mode). Alternatively, you can build individual packages by selecting the "Pack" command from the context menu for the corresponding project in Visual Studio.

\todo{can i prevent dotnet pack from producing errors?}
\todo{is there a way to pack all from VS?}

\end{enumerate}

\subsection{Installation}

To use CSharpE, install the CSharpE.Transform.VisualStudio extension from the Visual Studio Marketplace: \url{https://marketplace.visualstudio.com/items?itemName=svick.CSharpE-Transform-VisualStudio}. All following sections assume that this extension is already installed.

\subsection{Usage in example application}

An example application is available in the \cs{example} directory. It demonstrates the usage of three extensions:

\begin{enumerate}
\item The Logging extension adds logging of method name and arguments at the start of every method.

\item The Actor extension turns the method \cs{JsonProcessor.PrintAuthors} into an \cs{async} method that can be accessed only from one thread at a time.

Notice that calling this method from the \cs{Main} method requires the use of the \cs{await} keyword and that QuickInfo displays its return type as \cs{Task}, even though the return type in the source code is \cs{void}.

\item The Json extension acts as a type provider for JSON data. The attribute on the \cs{Authors}  class contains a JSON string that serves as a template.

The design-time version of the extension generates classes based on the structure of this template inside the \cs{Authors} class, along with a static \cs{Parse} method. Together, they allow the code that accesses them in the method \cs{JsonProcessor.PrintAuthors} to work.

The build-time version of this extension transforms calls to the \cs{Parse} method and to properties on the generated types into uses of the Json.NET library.
\end{enumerate}

\todo{move some of this explanation to section Example extensions?}

Build-time version of the generated code can be seen by invoking "Step Over" (F10) to start debugging and then immediately invoking "Stopp Debugging" (Shift+F5). This opens the version of \cs{Program.cs} generated by CSharpE from the \cs{obj\CSharpE} directory.

\todo{"immediatelly" probably won't work, explain it has to be done after code builds and starts}

\subsection{Usage in new project}

To use CSharpE in a new project, follow these steps:

\begin{enumerate}
\item Create a new .NET Core project in Visual Studio.

\todo{Clarify that .Net Framework projects work too, but have to use new-style csproj?}

\item Add reference to the NuGet package \cs{CSharpE.Transform.MSBuild}.

\item Add reference to one or more NuGet packages that contain CSharpE extensions. This can be a custom package (see the next section), or one of the following example extensions:

\begin{itemize}
\item \cs{CSharpE.Actor}
\item \cs{CSharpE.Json}
\item \cs{CSharpE.Logging}
\item \cs{CSharpE.Record}
\end{itemize}
\end{enumerate}

\subsection{Creating an extension}

\begin{enumerate}
\item Create a new .NET Standard 2.0 library in Visual Studio.

\item Add reference to the NuGet package \cs{CSharpE.Transform}.

\item Create a \cs{public} class that implements the \cs{ITransformation} interface.

The recommended approach is to do this by inheriting from one of the following \cs{abstract} classes: \cs{Transformation}, \cs{SimpleTransformation} or \cs{BuildTimeTransformation}. For more details on the available \acp{API}, see Sections~\ref{syntax-design} and \ref{transform-design}.

\item Create any other types required for the transformation, such as attributes for attribute-based transformations.

\item Implement the desired behavior of the extension.

\item Pack the project as a NuGet package and publish it to a configured NuGet package source. such as NuGet.org or a local directory.

\item To use the extension, follow the steps from the previous section.
\end{enumerate}