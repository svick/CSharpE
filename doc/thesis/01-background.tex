\chapter{Background}

The .NET ecosystem is composed of programming languages (including C\#, F\# and \acl{VB}), \cite{tour} .NET implementations (including .NET Framework, .NET Core and Mono), \cite{implementations} class libraries, commonly distributed through the NuGet package manager, and tooling, including command-line tools and tools integrated into code editors and \acp{IDE}.

What unifies all these components is the \ac{CLI}, \cite{ecma-335} which specifies binary file format for "assemblies". These contain compiled .NET code in the form of \ac{IL} and also metadata associated with this code.
%also known as \acf{CIL} or \acf{MSIL}

\medskip

The C\# language \cite{csharp-spec} is an object-oriented programming language which is part of the .NET ecosystem.
The C\# compiler, code named "Roslyn", \cite{roslyn} compiles C\# source code into a .NET assembly. The compiler can also be used as a class library, which exposes types for programmatically manipulating C\# source code.

\medskip

An assembly, produced by the C\# compiler or in some other way, can be executed on a .NET implementation. Each .NET implementation contains a runtime, which is responsible for executing code, and a base class library, which contains basic types used by .NET programs.

Runtimes of .NET implementations are usually using \iac{JIT} compiler, which converts the \ac{IL} for each method into machine code specific for the current instruction set just before executing that method for the first time. % mention AOT?

\medskip

In the .NET ecosystem, class libraries, which are just .NET assemblies, are commonly distributed thorough the NuGet package manager, \cite{nuget} because it makes using those libraries easier.
And while NuGet is primarily used for regular libraries, which are directly used by programmers in their source code, it can also be used for various kinds of special libraries, such as add-ins for general metaprogramming systems like Fody, or Roslyn analyzers for detecting issues with source code (more on these in section \ref{metaprogramming}).

\medskip

The C\# language contains some basic extensibility features itself, namely virtual methods and delegates.
But for more advanced use cases, it is necessary ot manipulate code in some form and the .NET ecosystem has various approaches to achieve that, including those that manipulate C\# code, those that manipulate \ac{IL} and those that use a custom model for representing code. Some of these approaches will be described in following sections.

\section{Example}

To demonstrate various code generation approaches, a running example of generating a simple entity class with a set of properties, a constructor and an implementation of the \cs{IComparable<T>} interface will be used.

For example, for an entity named \cs{Person} with properties \cs{Name} of type \cs{string} and \cs{Age} of type \cs{int}, the generated code should be similar to the following:

\codefile{samples/Core/Person.cs}{csharp}{}

\section{Manipulating C\# source code}

\subsection{\acs{T4}}

\ac{T4} is a tool for generating text by interspersing snippets of the text to generate with fragments of C\# code to control how the text is generated. The resulting text can be in any language, including C\#.

\ac{T4} does not have any special way of accessing other source code, which makes it most suitable for generating code based on external data. Its text-based nature gives it flexibility, but also makes using it fairly hard, since generating C\# code effectively requires writing two interleaved programs, without any help from the \ac{IDE}, because \ac{T4} integration into Visual Studio is very limited.

\subsubsection{Example}

To generate entities, as required by the running example, the \ac{T4} code could look like the following:

\codefile{samples/T4/Entities.tt}{text}{}

The code shown above highlights another issue with \ac{T4}: indentation. It is hard to keep indentation of both the generated code and the generating code consistent, especially since any whitespace outside of \ac{T4} tags will be included in the output.

\subsection{CodeDOM}

\ac{CodeDOM} is a library for generating source code by using a language-independent object model. It is fairly easy to use, but is limited in what language features it supports, due to its language-independent nature and due to it not being updated since .NET Framework 2.0. Some of these limitations can be worked around by using string-based "snippet" objects, but using them means negating the advantages that CodeDOM has. Some examples of features it does not support are declaring \cs{static} classes, LINQ query expressions or declaring auto-implemented properties.

\subsubsection{Example}

To generate entities for the running example using CodeDOM, the code could look like the following:

\todo{Why are only some types highlighted?}

\inputminted[firstline=14,lastline=108]{csharp}{samples/CodeDOM/Program.cs}

Especially notice that the above code has to manually generate fields for properties, because CodeDOM does not support auto-implemented properties.

\subsection{Roslyn}

\todo{mention IOperation (?) and SyntaxGenerator (?)}

\section{Manipulating \acs{IL}}

\todo{Reflection.Emit, System.Reflection.Metadata, Cecil}

\section{Other approaches}

\todo{Expression trees}

\section{Metaprogramming systems (?)}
\label{metaprogramming}

\todo{PostSharp, Fody, others}