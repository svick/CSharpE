\chapter{Background}

The .NET ecosystem is composed of programming languages (including C\#, F\# and \acl{VB}), \cite{tour} .NET implementations (including .NET Framework, .NET Core and Mono), \cite{implementations} class libraries, commonly distributed through the NuGet package manager, and tooling, including command-line tools and tools integrated into code editors and \acp{IDE}.

What unifies all these components is the \ac{CLI}, \cite{ecma-335} which specifies binary file format for "assemblies". These contain compiled .NET code in the form of \ac{IL} and also metadata associated with this code.
%also known as \acf{CIL} or \acf{MSIL}

\medskip

The C\# language \cite{csharp-spec} is an object-oriented programming language which is part of the .NET ecosystem.
The C\# compiler, code named "Roslyn", \cite{roslyn} compiles C\# source code into a .NET assembly. The compiler can also be used as a class library, which exposes types for programmatically manipulating C\# source code.

\medskip

An assembly, produced by the C\# compiler or in some other way, can be executed on a .NET implementation. Each .NET implementation contains a runtime, which is responsible for executing code, and a base class library, which contains basic types used by .NET programs.

Runtimes of .NET implementations are usually using \iac{JIT} compiler, which converts the \ac{IL} for each method into machine code specific for the current instruction set just before executing that method for the first time.

\todo{something about libraries (incl. NuGet), extensibility and metaprogramming}

\section{Manipulating C\# source code}

\subsection{CodeDOM}

\subsection{Roslyn}

\section{Manipulating \acs{IL}}

\todo{Reflection.Emit, System.Reflection.Metadata, Cecil}

\section{Metaprogramming systems (?)}

\todo{PostSharp, Fody, others}