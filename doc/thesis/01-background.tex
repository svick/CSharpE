\chapter{Background}

The .NET ecosystem is composed of programming languages (including C\#, F\# and \acl{VB}), \cite{tour} .NET implementations (including .NET Framework, .NET Core and Mono), \cite{implementations} class libraries, commonly distributed through the NuGet package manager, and tooling, including command-line tools and tools integrated into code editors and \acp{IDE}.

What unifies all these components is the \ac{CLI}, \cite{ecma-335} which specifies binary file format for "assemblies". These contain compiled .NET code in the form of \ac{IL} and also metadata associated with this code.
%also known as \acf{CIL} or \acf{MSIL}

\medskip

The C\# language \cite{csharp-spec} is an object-oriented programming language which is part of the .NET ecosystem.
The C\# compiler, code named "Roslyn", \cite{roslyn} compiles C\# source code into a .NET assembly. The compiler can also be used as a class library, which exposes types for programmatically manipulating C\# source code.

\medskip

An assembly, produced by the C\# compiler or in some other way, can be executed on a .NET implementation. Each .NET implementation contains a runtime, which is responsible for executing code, and a base class library, which contains basic types used by .NET programs.

Runtimes of .NET implementations are usually using \iac{JIT} compiler, which converts the \ac{IL} for each method into machine code specific for the current instruction set just before executing that method for the first time. % mention AOT?

\medskip

In the .NET ecosystem, class libraries, which are just .NET assemblies, are commonly distributed thorough the NuGet package manager, \cite{nuget} because it makes using those libraries easier.
And while NuGet is primarily used for regular libraries, which are directly used by programmers in their source code, it can also be used for various kinds of special libraries, such as add-ins for general metaprogramming systems like Fody (more on these in section \ref{metaprogramming}), or Roslyn analyzers for detecting issues with source code.

\medskip

The C\# language contains some basic extensibility features itself, namely virtual methods and delegates.
But for more advanced use cases, it is necessary ot manipulate code in some form and the .NET ecosystem has various approaches to achieve that, including those that manipulate C\# code, those that manipulate \ac{IL} and those that use a custom model for representing code. Some of these approaches will be described in following sections.

\section{Example}

To demonstrate various code generation approaches, a running example of generating a simple entity class with a set of properties and having \cs{IEquatable<T>} as an implemented interface will be used. (To make the examples shorter, the \cs{Equals} method required to properly implement the interface will not be included.)

For example, for an entity named \cs{Person} with properties \cs{Name} of type \cs{string} and \cs{Age} of type \cs{int}, the generated code should be similar to the one in Listing~\ref{example-result}.

\begin{listing}
\inputminted{csharp}{samples/Core/Person.cs}
\caption{Running example result}
\label{example-result}
\end{listing}

If an approach supports transforming C\# code, not just generating it, the example will instead be transforming simple classes containing only fields with the right types and names into the form above. Such transformation is too simple to be useful in practice, but it is sufficient as a demonstration.

\section{Manipulating C\# source code}

This section describes various approaches for generating and transforming C\# source code. The resulting code then needs to be compiled by the C\# compiler, as usual.

\subsection{\acs{T4}}

\ac{T4} \cite{T4} is a tool for generating text by interspersing snippets of the text to generate with fragments of C\# code to control how the text is generated. The resulting text can be in any language, including C\#.

\ac{T4} does not have any special way of accessing other source code, which makes it most suitable for generating code based on external data. Its text-based nature gives it flexibility, but also makes using it fairly hard, since generating C\# code effectively requires writing two interleaved programs, without any help from the \ac{IDE}, because \ac{T4} integration into Visual Studio is very limited.

\subsubsection{Example}

The code to generate entities using \ac{T4}, as required by the running example, can be seen in Listing~\ref{t4-example}.

\begin{listing}
\inputminted{text}{samples/T4/Entities.tt}
\caption{\ac{T4} example}
\label{t4-example}
\end{listing}

The code highlights another issue with \ac{T4}: indentation. It is hard to keep indentation of both the generated code and the generating code consistent, especially since any whitespace outside of \ac{T4} tags will be included in the output.

\subsection{CodeDOM}

\ac{CodeDOM} \cite{codedom} is a library for generating source code by using a language-independent object model. It is fairly easy to use, but is limited in what language features it supports, due to its language-independent nature and due to it not being updated since .NET Framework 2.0. Some of these limitations can be worked around by using string-based "snippet" objects, but using them means negating the advantages that CodeDOM has. Some examples of features it does not support are declaring \cs{static} classes, LINQ query expressions or declaring auto-implemented properties.

\subsubsection{Example}

The code to generate entities for the running example using CodeDOM can be seen in Listing~\ref{codedom-example}.

\begin{listing}
\inputminted[firstline=14,lastline=61]{csharp}{samples/CodeDOM/Program.cs}
\caption{CodeDOM example}
\label{codedom-example}
\end{listing}

Especially notice that the code has to manually generate backing fields for properties, because CodeDOM does not support auto-implemented properties.

\subsection{Roslyn}

Roslyn \cite{roslyn} is the C\# (and Visual Basic .NET) compiler, which can also be used as a library for manipulating C\# code, including parsing, transformation and generation. Its object model was primarily designed to be used in the Visual Studio \ac{IDE}, which is why it is very detailed (so it can accurately represent any source code, including erroneous or incomplete code) and also immutable (so that multiple \ac{IDE} services can operate on the same model).

Roslyn contains several related \acp{API} for manipulating source code, each useful in different situations:

\begin{itemize}
\item The \cs{SyntaxTree} \ac{API} forms the basis of Roslyn and represents only syntactic information about code. This means it can be used for just a single source file and makes it very efficient, especially when creating the \cs{SyntaxTree} for code that contains only a small change relative to another \cs{SyntaxTree}.

On the other hand, no semantic information is available from this \ac{API}, so for example for the expression \cs{F(A.B)}, it is not possible to determine whether \cs{F} is a method or a delegate, whether \cs{A} refers to a type or a variable, or whether \cs{B} is a field, a property, or a method group.

The \cs{SyntaxFactory} class can be used to create new nodes for this \cs{API}.

\item The \cs{SemanticModel} class can be used to answer semantic questions about some part of a \cs{SyntaxTree}.

The disadvantage of using this class is that it requires a full compilation, which includes all files in a project and also all of its dependencies. It is also less efficient, especially when a change is made.

The \cs{SemanticModel} class surfaces semantic information in two forms:

\begin{itemize}
\item The \cs{ISymbol} \ac{API} can be used to get semantic information about members declared or referenced by a piece of code.

For example, for the expression \cs{Console.WriteLine(42)}, this \ac{API} could return the symbol for the \cs{System.Console.WriteLine(int)} method. That symbol could then be used to find out more semantic information about that method, like the assembly it is contained in.

\item The \cs{IOperation} \ac{API} is an alternative representation of statements and expressions as a language-independent abstract syntax tree. It includes semantic information in the form of \cs{ISymbol} objects.

\end{itemize}

\item The \cs{SyntaxGenerator} class offers an alternative, language-independent way of generating Roslyn syntax nodes. It is part of the Workspaces layer of Roslyn, which means that using it on its own requires some additional setup. \cs{SyntaxGenerator} has a more semantic view of code, which can be easier than generating the exact syntax using \cs{SyntaxFactory}.

Since \cs{SyntaxGenerator} is language-independent, it is using the \cs{SyntaxNode} type  in its \ac{API} to represent any kind of syntax node, because \cs{SyntaxNode} is the common base class for syntax node types in different languages. This approach makes \cs{SyntaxGenerator} less type-safe compared with \cs{SyntaxFactory}, which uses specific \cs{SyntaxNode}-derived types in its \ac{API}.
\end{itemize}

\subsubsection{Example}

The code to transform entities for the running example using the \cs{SyntaxTree} \ac{API} can be seen in Listing~\ref{syntaxtree-example}.

Note that this code is heavily using \cs{SyntaxFactory} to create new syntax nodes, but that type is not visible due to use of \cs{using static}, to make the code more succinct.

\begin{listing}
\inputminted[firstline=15,lastline=50]{csharp}{samples/Roslyn/Program.cs}
\caption{Roslyn \cs{SyntaxTree} example}
\label{syntaxtree-example}
\end{listing}

Also notice how immutability makes transforming code harder by requiring the use of methods such as \cs{ReplaceNodes} and how the level of detail leads to very verbose code, in some cases requiring even the specification of individual semicolons.

\medskip

Code to transform entities for the running example using \cs{SyntaxGenerator} can be seen in Listing~\ref{syntaxgenerator-example}.

\begin{listing}
\inputminted[firstline=17,lastline=73]{csharp}{samples/Roslyn.SyntaxGenerator/Program.cs}
\caption{Roslyn \cs{SyntaxGenerator} example}
\label{syntaxgenerator-example}
\end{listing}

Note that \cs{SyntaxGenerator} does not support auto-implemented properties, so the example has to create properties with backing fields.

Also note that \cs{SyntaxGenerator} generates overly verbose code (for example, \cs{global::System.IEquatable<global::Person>}), but this is counterweighted by being able to use another Workspace-layer service, \cs{Simplifier}, to make the code simpler.

\todo{More obscure libraries like RoslynDOM}

\section{Manipulating \acs{IL}}

This section describes various libraries that can be used for generating and transforming \ac{IL} code.

Since \ac{IL} is primarily meant to be produced by compilers and consumed by .NET runtimes, it is not a very convenient language for programmers.

\subsection{System.Reflection.Emit}

System.Reflection.Emit \cite{refemit} is a library that can be used to generate an assembly at the \ac{IL} level in memory, and then either directly execute code from that assembly or save the assembly to disk.

\subsubsection{Example}

Code to generate entities for the running example using Reflection.Emit can be seen in Listing~\ref{refemit-example}.

\begin{listing}
\inputminted[firstline=12,lastline=64]{csharp}{samples/ReflectionEmit/Program.cs}
\caption{System.Reflection.Emit example}
\label{refemit-example}
\end{listing}

Notice how generating properties requires understanding that they are represented using special methods and that generating bodies for those methods requires understanding the \ac{IL} stack machine.

\subsection{Mono.Cecil}
\label{cecil}

Mono.Cecil \cite{cecil} is a library that can be used for generating and transforming assemblies. It was written as part of the Mono project.

The main difference between Cecil and Reflection.Emit is that Cecil can be used to read assemblies, including their \ac{IL}, and to modify them, while Reflection.Emit can only be used to create brand new assemblies. Another difference is that Cecil has its own type system, independent from the type system of the .NET runtime. This means that it can be used for example to work with assemblies that target a newer version of .NET than the currently executing one, or to work with assemblies that target an incompatible .NET implementation.

\medskip

Cecil does not have an example of its usage shown here. While its \acp{API} is different from Reflection.Emit, those differences are not relevant to this work.

\section{Other approaches}

While with the approaches mentioned in previous sections, then generated code (either C\# or \ac{IL}) fairly closely corresponds to the generating code, other options are possible. This section describes one such library.

\subsection{Expression trees}
\label{expression-trees}

Expression trees, \cite{expression-trees} contained in the System.Linq.Expressions namespace, offer another representation of code.

Expression trees were first introduced in .NET Framework 3.5, to support translating of \ac{LINQ} queries to existing query languages, such as \ac{SQL}. This initial version included only expression-like constructs and the C\# language supported compiling of expression lambdas to code that creates the corresponding expression tree.

In .NET Framework 4.0, expression trees were expanded with statement-like constructs, such as blocks, assignments or loops, to support the \ac{DLR}. The result is a "language" that is still expression-based, which means that even constructs such as blocks or loops can have a result. This is somewhat similar to functional languages such as F\#, which also do not differentiate between statements and expressions. The C\# language was not updated to support the new constructs when translating lambdas to expression trees.

An expression tree can be inspected, often in order to be translated to some query language, or it can be executed. Depending on circumstances, executing expression trees is either done by using an interpreter, or by compiling them to \ac{IL} using Reflection.Emit and then executing the result.

Expression trees can only represent expressions and statements, not types or their members, which limits their usefulness when generating code. This also means the running example is not applicable to expression trees.

\section{Metaprogramming systems}
\label{metaprogramming}

\todo{More obscure tools, especially those that work with C\#}

\subsection{PostSharp}

PostSharp \cite{postsharp} is a commercial tool for transforming built assemblies at the \ac{IL} level. It focuses on \ac{AOP}, which is the idea that cross-cutting concerns (related pieces of code that are spread over the program, such as logging or code related to thread safety) should be specified separately from the rest of the code.

An aspect is applied to the target program element, such as a method, by attaching a specific attribute to it. The attribute can also be attached to a container, such as a type or an assembly, which applies the aspect to all relevant program elements in that container. This is called "attribute multicasting".

PostSharp includes many built-in aspects and also allows specifying custom aspects. Custom aspects work by calling a user-defined method at a specific point, such as at the start of a method or before another method is called. The user-defined method can also be provided additional information about the modified method, such as its name or arguments.

This approach makes writing custom aspects easy, but it also means they are limited in what they can do and can have some performance penalty (due to allocation of the object that contains information about the modified method).

Like other tools that work at the \ac{IL} level, PostSharp is also limited when it comes to changes to the shape of types, because any changes it makes will not be visible at compile time, only at runtime.

\subsection{Fody}

Fody \cite{fody} is an open-source tool for transforming the \ac{IL} of assemblies. There are many published "add-ins" for Fody, usually distributed through NuGet, and custom add-ins can be written by modifying the assembly using the Mono.Cecil \ac{API} (see Section \ref{cecil}). This makes custom add-ins hard to write, but it means they can perform any transformation. Though the limitations of \ac{IL}-based tools still apply: changes to the shape of type will not be visible at compile time.

\subsection{F\# type providers}

The F\# language contains a feature called "type providers", \cite{type-providers} meant for easier access to data sources. Type providers generate types at design-time (i.e. while editing code in an \ac{IDE}) based on their input parameters and on usage of the generated types. These types can be either regular types that still exist after compilation (type providers using this approach are called "generative") or their use can be transformed into some other code ("erased type providers").

Type providers use "code quotations" to express code to execute. Code quotations serve a similar purpose in F\# as expression trees do in C\#.